\documentclass{resume} % Use the custom resume.cls style

\usepackage[left=0.4 in,top=0.4in,right=0.4 in,bottom=0.4in]{geometry} % Document margins
\newcommand{\tab}[1]{\hspace{.2667\textwidth}\rlap{#1}} 
\newcommand{\itab}[1]{\hspace{0em}\rlap{#1}}
\name{Georgios Christou} % Your name
\address{\href{mailto:giorgos.christou@protonmail.com}{giorgos.christou@protonmail.com} \\ \href{https://www.linkedin.com/in/georgios-christou-5b3342244/}{LinkedIn} \\ \href{https://github.com/GiorgosChr}{GitHub}}  %

\begin{document}
        \vspace{-1em}
        \begin{rSection}{Education}
                {\bf PhD Particle Physics}, The University of Edinburgh, Edinburgh, Scotland \hfill {Sept 2024 - Present}\\

                \vspace{-1.3em}

                {\bf MSc Particle and Nuclear Physics}, The University of Edinburgh, Edinburgh, Scotland \hfill {Sept 2023 - Aug 2024}\\
                Graduated with A3 Distinction, $1^{st}$ in class, GPA: $75/100$
                
                \vspace{-0.3em}

                {\bf BSc Physics}, University of Cyprus, Nicosia, Cyprus \hfill {Sept 2019 - Jun 2023}\\
                Graduated with Excellence, $1^{st}$ in class, GPA: $8.66/10$
                
        %         {\bf High School Diploma}, Lyceum Makariou III, Larnaca, Cyprus \hfill {Sept 2015 - Jun 2018}\\
        % Graduated with Excellence, GPA: $19.22/20$
        \end{rSection}

        \vspace{-0.5em}

        \begin{rSection}{RESEARCH EXPERIENCE}
                \textbf{PhD Project}, The University of Edinburgh, United Kingdom \hfill Sept 2024 - Present\\
                % {\it Proton structure and light quark Yukawa couplings}\hfill \textit{University of Edinburgh}
                \vspace{-1.5em}
                \begin{itemize}
                        \itemsep -2pt {}
                        \item Developed machine learning regressors to improve separation of signal from background in hypothesis testing searches.
                        \item Applied advanced optimization techniques to boost search sensitivity, achieving up to 25\% improvement.
                        \item Integrated ML pipelines into high-energy physics workflows, demonstrating scalable impact on large datasets.
                \end{itemize}

                \textbf{MSc Thesis}, The University of Edinburgh, United Kingdom \hfill Nov 2023 - Aug 2024\\
                % {\it Proton structure and light quark Yukawa couplings}\hfill \textit{University of Edinburgh}
                \vspace{-1.5em}
                \begin{itemize}
                        \itemsep -2pt {}
                        \item Built and compared machine learning models to improve detection of subtle patterns in complex scientific data.
                        \item Combined model outputs with statistical analysis techniques to set upper limits on key parameters with enhanced precision.
                        \item Achieved results on par with cutting-edge benchmarks, showcasing the potential of ML in high-impact data analysis.
                \end{itemize}

                \textbf{CERN Summer Student Programme 2023}, Switzerland  \hfill Jun 2023 - Aug 2023\\
                % {\it CP asymmetries in charm decays}
                % \hfill \textit{LHCb Collaboration}
                \vspace{-1.5em}
                \begin{itemize}
                        \itemsep -2pt {}
                        \item Designed and implemented a C++ based statistical re-weighting algorithm to enhance the precision of asymmetry measurements.
                        \item Benchmarked performance against previous methods, demonstrating measurable improvements in accuracy and reliability.
                \end{itemize}

                \textbf{BSc Thesis and Undergaduate Internship}, University of Cyprus, Cyprus \hfill May 2022 - May 2023\\
                % {\it Baryon Spectrum using Lattice QCD}\hfill \textit{University of Cyprus}
                \vspace{-1.5em}
                \begin{itemize}
                        \itemsep -2pt {} 
                        \item Applied advanced statistical techniques (multi-state fits, model averaging) to analyze lattice QCD data and extract baryon masses at the physical pion mass.
                        \item Devised and implemented novel fitting strategies, achieving results in strong agreement with experimental and theoretical benchmarks.
                        \item Published findings in Physical Review D, marking the first calculation at the physical point for this spectrum.
                \end{itemize}
        \end{rSection} 

        \vspace{-0.5em}

        \begin{rSection}{AWARDS \& ACHIEVEMENTS}
                {\bf Class Medal Award for MSc in Particle and Nuclear Physics}, The University of Edinburgh \hfill Nov 2024\\
                Awarded for the excellent performance in the MSc in Particle and Nuclear Physics

                {\bf Valedictorian in the Department of Physics}, University of Cyprus \hfill Jun 2023\\
                Awarded to the student with the highest GPA of the department
        \end{rSection} 

        % \vspace{-0.5em}

        % \begin{rSection}{TEACHING EXPERIENCE}
        %         \textbf{Teaching Assistant} \hfill Sept 2024 - Present\\
        %         {\it The University of Edinburgh}
        %         \begin{itemize}
        %                 \itemsep -3pt {} 
        %                 \item Assisting students  for the courses \href{http://www.drps.ed.ac.uk/24-25/dpt/cxpgph11105.htm}{DAML (Postgraduate course)}, \href{http://www.drps.ed.ac.uk/24-25/dpt/cxphys09057.htm}{Computer Modelling (Undergaduate course)}
        %         \end{itemize}
        % \end{rSection}

        % \vspace{-0.5em}

        % \begin{rSection}{PUBLICATIONS}
        %         \begin{itemize}
        %                 \item C.~Alexandrou, S.~Bacchio, \textbf{G.~Christou}, and J.~Finkenrath, ``Low-lying baryon masses using twisted mass fermions ensembles at the physical pion mass,'' \textit{Phys. Rev. D} \textbf{108}, 094510 (2023), \href{https://arxiv.org/abs/2309.04401}{arXiv:2309.04401}.
        %         \end{itemize}
        % \end{rSection}
        
        \vspace{-0.5em}

        \begin{rSection}{SKILLS}
                \begin{itemize}
                        \itemsep -2pt {} 
                        \item \textbf{Programming}: Python, C++, Bash/Shell, Fortran, Mathematica
                        \item \textbf{Languages}: Greek (Native), English (IELTS Score: 8, Level: C1), French (Beginner)
                        \item \textbf{Technical}: Git, \LaTeX, Linux, Unix, Machine Learning (scikit-learn, TensorFlow, PyTorch), Data Analysis (NumPy, Pandas), Publication-grade Data Visualization (MatPlotLib, Seaborn), Database Knowledge (HDF5)
                        % \item \textbf{Soft Skills}: Communication (technical and non-technical), Collaboration \& Teamwork, Time Management, Problem Solving \& Critical Thinking
                \end{itemize}
        \end{rSection}

        \bigbreak

        % \begin{rSection}{OTHER}
        %         {\bf Cypriot National Guard Military Service}: Cyprus, 14 Months \hfill {Jul 2018 - Sept 2019}\\
        %         Rank: Private
        % \end{rSection}

        \bigbreak

        % \begin{rSection}{INTERESTS} 
        %         My hobbies include photography and especially wide-field and deep-sky astrophotography, as well as creating time-lapse videos.
        %         Moreover I enjoy playing guitar, listening to music and reading books.
        %         Another passion of mine is creating programs to solve numerical problems in physics.
        % \end{rSection}

\end{document}
