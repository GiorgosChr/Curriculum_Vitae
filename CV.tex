\documentclass{resume} % Use the custom resume.cls style

\usepackage[left=0.4 in,top=0.4in,right=0.4 in,bottom=0.4in]{geometry} % Document margins
\newcommand{\tab}[1]{\hspace{.2667\textwidth}\rlap{#1}} 
\newcommand{\itab}[1]{\hspace{0em}\rlap{#1}}
\name{Georgios Christou} % Your name
\address{\href{mailto:giorgos.christou@protonmail.com}{giorgos.christou@protonmail.com} \\ \href{https://www.linkedin.com/in/georgios-christou-5b3342244/}{LinkedIn} \\ \href{https://github.com/GiorgosChr}{Github}}  %

\begin{document}


\begin{rSection}{Education}

{\bf BSc, Physics}, University of Cyprus \hfill {2019-2023 (Expected)}\\
GPA: 8.59/10

{\bf High School Diploma}, Archbishop Makarios III Lyceum \hfill {2015 - 2018}\\
GPA: 19.22/20


\end{rSection}


\begin{rSection}{RESEARCH EXPERIENCE}

\textbf{Wave function of the universe for the Starobinsky inflationary model} \hfill Jun 2021 - Aug 2021\\
Undergdraduate Internship, Project Supervisor:  \href{https://ucyweb.ucy.ac.cy/dir/en/component/comprofiler/userprofile/nick}{Prof. Nicolaos Toumbas}\hfill \textit{University of Cyprus}
 \begin{itemize}
    \itemsep -3pt {} 
     \item The main purpose of this project was to see whether initial conditions favouring inflation are probable.
     \item We approximated the Starobinsky potential as a step function and we used the WKB approximation in the semiclassical regime in order to find the wave function for various values of the inflaton field.
    \item Using appropriate boundary conditions we constructed the quantum probability density distribution for this inflationary model.
 \end{itemize}
 
\textbf{Baryon spectrum using lattice QCD simulations} \hfill May 2022 - Jun 2022\\
Undergdraduate Internship, Project Supervisor:  \href{https://www.cyi.ac.cy/index.php/castorc/about-the-center/castorc-our-people/itemlist/user/99-constantia-alexandrou.html}{Prof. Constantia Alexandrou}\hfill \textit{University of Cyprus}
 \begin{itemize}
    \itemsep -3pt {} 
     \item Calculation of various baryon masses using correlator data generated from lattice QCD simulations.
     \item Using different computational methods we extracted the mass of each baryon from the effective mass by performing various fits for two ensembles.
 \end{itemize}

\textbf{Continuum limit of low-lying baryon masses} \hfill Sept 2022 - May 2023\\
BSc Thesis, Project Supervisor:  \href{https://www.cyi.ac.cy/index.php/castorc/about-the-center/castorc-our-people/itemlist/user/99-constantia-alexandrou.html}{Prof. Constantia Alexandrou}\hfill \textit{University of Cyprus}
 \begin{itemize}
    \itemsep -3pt {} 
     \item The thesis was a continuation of the previous project and the purpose was to complete the calculations for the baryon mass spectrum.
     \item Using three different ensembles with different lattice spacings and various computational techniques it was possible to calculate the baryon mass spectrum at the continuum limit and compare the results to the experimental values.
 \end{itemize}

\end{rSection} 


\begin{rSection}{AWARDS \& ACHIEVEMENTS}
\vspace{-1.25em}
\item \textbf{Grade 5 Music Theory}: {The Associated Board of the Royal Schools of Music. Grade: Distinction}
\item \textbf{Electric Guitar Degree}: {Grade: Excellent}
\end{rSection} 

\begin{rSection}{SKILLS} 
\item \textbf{Programming}: Fortran, Mathematica, C++, Python, ROOT, Bash/Shell
\item \textbf{Languages}: Greek (Native), English (IELTS Score: 8), French (Beginner)
\item \textbf{Technical}: Git, Github, \LaTeX, Linux, Unix


\end{rSection}

\begin{rSection}{INTERESTS} 
\begin{itemize}
   \item My hobbies include photography and especially wide-field and deep-sky astrophotography, as well as creating time-lapse videos. Moreover I enjoy playing guitar and listening to music. 
\end{itemize}


\end{rSection}


\end{document}

