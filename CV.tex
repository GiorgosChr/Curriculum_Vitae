\documentclass{resume} % Use the custom resume.cls style

\usepackage[left=0.4 in,top=0.4in,right=0.4 in,bottom=0.4in]{geometry} % Document margins
\newcommand{\tab}[1]{\hspace{.2667\textwidth}\rlap{#1}} 
\newcommand{\itab}[1]{\hspace{0em}\rlap{#1}}
\name{Georgios Christou} % Your name
\address{\href{mailto:giorgos.christou@protonmail.com}{giorgos.christou@protonmail.com} \\ \href{https://www.linkedin.com/in/georgios-christou-5b3342244/}{LinkedIn} \\ \href{https://github.com/GiorgosChr}{Github}}  %

\begin{document}
\begin{rSection}{Education}
{\bf M.Sc Particle and Nuclear Physics}, University of Edinburgh, Edinburgh, Scotland \hfill {Sept 2023 - Present}

{\bf B.Sc Physics}, University of Cyprus, Nicosia, Cyprus \hfill {Sept 2019 - Jun 2023}\\
Graduated with Excellence, $1^{st}$ in class, GPA: $8.66/10$

{\bf High School Diploma (Apolytirio)}, Lyceum Makariou III, Larnaca, Cyprus \hfill {Sept 2015 - Jun 2018}\\
Graduated with Excellence, GPA: $19.22/20$
\end{rSection}

\bigbreak

\begin{rSection}{RESEARCH EXPERIENCE}
\textbf{CERN Summer Student Programme 2023} \hfill Jun 2023 - Aug 2023\\
Internship, Project Supervisors: \href{https://www.unibo.it/sitoweb/angelo.carbone/en}{Prof. Angelo Carbone}, \href{https://www.ph.ed.ac.uk/people/federico-betti}{Dr. Federico Betti}
\hfill \textit{LHCb Collaboration}
\begin{itemize}
\itemsep -3pt {}
\item Study of $CP$ asymmetries in charm decays at the \href{https://lhcb.web.cern.ch/}{LHCb Collaboration}.
\item Development of new kinematic weighting algorithm for the measurement of $CP$ asymmetries.
\item Implementation of RapidSim and RDataFrame to simulate and analyze data.
\end{itemize}

\textbf{Continuum limit of the low-lying baryon spectrum} \hfill Sept 2022 - May 2023\\
BSc Thesis, Project Supervisor:  \href{https://www.cyi.ac.cy/index.php/castorc/about-the-center/castorc-our-people/itemlist/user/99-constantia-alexandrou.html}{Prof. Constantia Alexandrou}\hfill \textit{University of Cyprus}
\begin{itemize}
\itemsep -3pt {} 
\item The thesis was a continuation of the previous project and the purpose was to complete the calculations for the baryon mass spectrum.
\item Became familiar with the environment of exascale computers.
\item Implementation of model averaging for bias elimination.
\item Using three different ensembles with different lattice spacings and various computational techniques it was possible to calculate the baryon mass spectrum at the continuum limit and compare the results to the experimental values.
\item Prediction of previously unmeasured masses of doubly- and triply-charmed baryons.
\end{itemize}

\textbf{Low-lying baryon spectrum using lattice QCD simulations} \hfill May 2022 - Jun 2022\\
Undergdraduate Internship, Project Supervisor:  \href{https://www.cyi.ac.cy/index.php/castorc/about-the-center/castorc-our-people/itemlist/user/99-constantia-alexandrou.html}{Prof. Constantia Alexandrou}\hfill \textit{University of Cyprus}
\begin{itemize}
\itemsep -3pt {} 
\item Calculation of various baryon masses using correlator data generated from lattice QCD simulations.
\item Implementation of methods for evaluating the low-lying baryon spectrum at finite lattice spacing.
\end{itemize}

\textbf{Wave function of the universe for the Starobinsky inflationary model} \hfill Jun 2021 - Aug 2021\\
Undergdraduate Internship, Project Supervisor:  \href{https://www.ucy.ac.cy/directory/en/profile/nick}{Prof. Nicolaos Toumbas}\hfill \textit{University of Cyprus}
\begin{itemize}
\itemsep -3pt {} 
\item The main purpose of this project was to see whether initial conditions favouring inflation are probable.
\item We approximated the Starobinsky potential as a step function and we used the WKB approximation in the semiclassical regime in order to find the wave function for various values of the inflaton field.
\item Using appropriate boundary conditions we constructed the quantum probability density distribution for this inflationary model.
\end{itemize}
\end{rSection} 

\bigbreak



\begin{rSection}{SKILLS}
\itemsep -3pt {} 
\item \textbf{Programming}: Fortran, Mathematica, C++, Python, ROOT, Bash/Shell
\item \textbf{Languages}: Greek (Native), English (IELTS Score: 8, Level: C1), French (Beginner)
\item \textbf{Technical}: Git, Github, \LaTeX, Linux, Unix
\end{rSection}

\bigbreak

\begin{rSection}{AWARDS \& ACHIEVEMENTS}
{\bf Valedictorian in the Department of Physics}, University of Cyprus \hfill Jun 2023\\
Awarded to the student with the highest GPA of the department

{\bf Grade 5 Music Theory}, The Associated Board of the Royal Schools of Music \hfill May 2019\\
Grade: Distinction

{\bf Electric Guitar Degree}, Musical Horizons Conservatory \hfill Nov 2017\\
Grade: Excellent
\end{rSection} 

\bigbreak

\begin{rSection}{OTHER}
{\bf Military Service}: Cyprus, 14 Months \hfill {Jul 2018 - Sept 2019}\\
Rank: Private
\end{rSection}

\bigbreak

\begin{rSection}{INTERESTS} 
My hobbies include photography and especially wide-field and deep-sky astrophotography, as well as creating time-lapse videos.
Moreover I enjoy playing guitar and listening to music.
Another passion of mine is creating programs to solve numerical problems in physics.
Lastly, I enjoy reading fantasy books.
\end{rSection}

\end{document}