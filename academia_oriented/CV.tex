\documentclass{resume} % Use the custom resume.cls style

\usepackage[left=0.4 in,top=0.4in,right=0.4 in,bottom=0.4in]{geometry} % Document margins
\newcommand{\tab}[1]{\hspace{.2667\textwidth}\rlap{#1}} 
\newcommand{\itab}[1]{\hspace{0em}\rlap{#1}}
\name{Georgios Christou} % Your name
\address{\href{mailto:giorgos.christou@protonmail.com}{giorgos.christou@protonmail.com} \\ \href{https://www.linkedin.com/in/georgios-christou-5b3342244/}{LinkedIn} \\ \href{https://github.com/GiorgosChr}{GitHub}}  %

\begin{document}

        \begin{rSection}{Education}
                {\bf PhD Particle Physics}, The University of Edinburgh, Edinburgh, Scotland \hfill {Sept 2024 - Present}\\

                {\bf MSc Particle and Nuclear Physics}, The University of Edinburgh, Edinburgh, Scotland \hfill {Sept 2023 - Aug 2024}\\
                Graduated with Distinction, $1^{st}$ in class, GPA: $75/100$
                
                {\bf BSc Physics}, University of Cyprus, Nicosia, Cyprus \hfill {Sept 2019 - Jun 2023}\\
                Graduated with Excellence, $1^{st}$ in class, GPA: $8.66/10$
                
                {\bf High School Diploma}, Lyceum Makariou III, Larnaca, Cyprus \hfill {Sept 2015 - Jun 2018}\\
        Graduated with Excellence, GPA: $19.22/20$
        \end{rSection}

        \bigbreak

        \begin{rSection}{RESEARCH EXPERIENCE}
                \textbf{MSc Thesis} \hfill Nov 2023 - Present\\
                {\it Proton structure and light quark Yukawa couplings}, Supervisor: \href{https://www.ph.ed.ac.uk/people/liza-mijovic}{Dr.\@ Liza Mijovi\'c} \hfill \textit{University of Edinburgh}
                \begin{itemize}
                        \item Usage of machine learning for classification of different Higgs boson production modes in the di-photon channel
                        \item Statistical analysis and interpretation of the results.
                        \item First implementation of a novel approach for measuring light quark Yukawa couplings based on the production modes using the di-photon kinematics.
                        \item Set stringent constrains on the light quark Yukawa couplings.
                \end{itemize}

                \textbf{CERN Summer Student Programme 2023} \hfill Jun 2023 - Aug 2023\\
                {\it CP asymmetries in charm decays}, Supervisors: \href{https://www.unibo.it/sitoweb/angelo.carbone/en}{Prof.\@ Angelo Carbone}, \href{https://www.ph.ed.ac.uk/people/federico-betti}{Dr.\@ Federico Betti}
                \hfill \textit{LHCb Collaboration}
                \begin{itemize}
                        \itemsep -3pt {}
                        \item Development of new kinematic weighting algorithm for the measurement of $CP$ asymmetries.
                        \item Implementation of RapidSim and Particle Gun to simulate data.\\ The project report is on \href{https://cds.cern.ch/record/2866568/}{CDS} and on \href{https://github.com/GiorgosChr/CERN_Summer_Student_Programme_2023}{GitHub}.
                \end{itemize}

                \textbf{BSc Thesis} \hfill Sept 2022 - May 2023\\
                {\it Baryon Spectrum using Lattice QCD}, Supervisor:  \href{https://www.cyi.ac.cy/index.php/castorc/about-the-center/castorc-our-people/itemlist/user/99-constantia-alexandrou.html}{Prof.\@ Constantia Alexandrou}\hfill \textit{University of Cyprus}
                \begin{itemize}
                        \itemsep -3pt {} 
                        \item The thesis was a continuation of the previous project and the purpose was to complete the calculations for the baryon mass spectrum.
                        \item The first ever calculation of the low-lying baryon spectrum at the continuum limit using exclusively physical point twisted mass fermion ensembles.
                        % \item Became familiar with the environment of exascale computers.
                        \item Calculation of the baryon mass spectrum at the continuum limit and comparison with experimental values.
                        \item Prediction of previously unmeasured low-lying masses of doubly- and triply-charmed baryons.
                \end{itemize}

                \textbf{Undergaduate Internship} \hfill May 2022 - Jun 2022\\
                {\it Baryon masses from Lattice QCD}, Supervisor:  \href{https://www.cyi.ac.cy/index.php/castorc/about-the-center/castorc-our-people/itemlist/user/99-constantia-alexandrou.html}{Prof.\@ Constantia Alexandrou}\hfill \textit{University of Cyprus}
                \begin{itemize}
                        \itemsep -3pt {} 
                        \item Calculation of various baryon masses using correlator data generated from lattice QCD simulations.
                        \item Implementation of methods for evaluating the low-lying baryon spectrum at finite lattice spacing.
                \end{itemize}

                \textbf{Undergaduate Internship} \hfill Jun 2021 - Aug 2021\\
                {\it Wheeler-DeWitt solution for Starobinsky potential}, Supervisor:  \href{https://www.ucy.ac.cy/directory/en/profile/nick}{Prof.\@ Nicolaos Toumbas}\hfill \textit{University of Cyprus}
                \begin{itemize}
                        \itemsep -3pt {} 
                        \item The main purpose of this project was to see whether initial conditions favouring inflation are probable.
                        \item We approximated the Starobinsky potential as a step function and we used the WKB approximation in the semiclassical regime in order to find the wave function for various values of the inflaton field.
                        \item Using appropriate boundary conditions we constructed the quantum probability density distribution for this inflationary model.
                \end{itemize}
        \end{rSection} 

        \bigbreak

        \begin{rSection}{TEACHING EXPERIENCE}
                \textbf{Teaching Assistant} \hfill Sept 2024 - Present\\
                {\it The University of Edinburgh}
                \begin{itemize}
                        \itemsep -3pt {} 
                        \item Assisting students with workshop problems and marking assignments on machine learning, simulations and statistical analysis for the courses \href{http://www.drps.ed.ac.uk/24-25/dpt/cxpgph11105.htm}{DAML}, \href{http://www.drps.ed.ac.uk/24-25/dpt/cxphys09057.htm}{Computer Modelling}
                \end{itemize}
        \end{rSection}

        \bigbreak

        \begin{rSection}{PUBLICATIONS}
                \begin{itemize}
                        \item A list of my publications can be found on my \href{https://inspirehep.net/authors/2313765}{INSPIRE} profile.
                \end{itemize}
        \end{rSection}

        \bigbreak

        \begin{rSection}{SKILLS}
                \begin{itemize}
                        \itemsep -3pt {} 
                        \item \textbf{Programming}: Python, C++, Bash/Shell, Fortran, Mathematica
                        \item \textbf{Languages}: Greek (Native), English (IELTS Score: 8, Level: C1), French (Beginner)
                        \item \textbf{Technical}: Git, GitHub, \LaTeX, Linux, Unix, Machine Learning (scikit-learn, TensorFlow, Keras), Data Analysis (NumPy, Pandas, Matplotlib)
                \end{itemize}
        \end{rSection}

        \bigbreak

        \begin{rSection}{AWARDS \& ACHIEVEMENTS}
                {\bf Class Medal Award for MSc in Particle and Nuclear Physics}, The University of Edinburgh \hfill Nov 2024\\
                Awarded for the excellent performance in the MSc in Particle and Nuclear Physics

                {\bf Valedictorian in the Department of Physics}, University of Cyprus \hfill Jun 2023\\
                Awarded to the student with the highest GPA of the department
                
                % {\bf Grade 5 Music Theory}, The Associated Board of the Royal Schools of Music \hfill May 2019\\
                % Grade: Distinction
                
                % {\bf Electric Guitar Degree}, Musical Horizons Conservatory \hfill Nov 2017\\
                % Grade: Excellent
        \end{rSection} 

        \bigbreak

        \begin{rSection}{OTHER}
                {\bf Cypriot National Guard Military Service}: Cyprus, 14 Months \hfill {Jul 2018 - Sept 2019}\\
                Rank: Private
        \end{rSection}

        \bigbreak

        % \begin{rSection}{INTERESTS} 
        %         My hobbies include photography and especially wide-field and deep-sky astrophotography, as well as creating time-lapse videos.
        %         Moreover I enjoy playing guitar, listening to music and reading books.
        %         Another passion of mine is creating programs to solve numerical problems in physics.
        % \end{rSection}

\end{document}
